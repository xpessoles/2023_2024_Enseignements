\documentclass[10pt,fleqn]{article} % Default font size and left-justified equations
\usepackage[%
    pdftitle={Exercices de SII},
    pdfauthor={Xavier Pessoles}]{hyperref}

\newcommand{\repRel}{../../..}
\input{\repRel/Style/packages}
\input{\repRel/Style/new_style}
\input{\repRel/Style/macros_SII}
\input{\repRel/Style/environment}
\usepackage{\repRel/Style/UPSTI_pedagogique}


\newcommand{\macrocomp}{macro_competences}
\newcommand{\comp}{competences}
\newcommand{\td}{fichier_td}
\newcommand{\repExo}{dossier}
\newcommand{\repStyle}{\repRel/Style}

\def\xxYCartouche{-2.25cm}
\def\xxYongletGarde{.5cm}
\def\xxYOnget{.9cm}


\def\xxnumchapitre{}
\def\xxchapitre{Résistance des Matériaux}
\def\xxtitreexo{Torseur de cohésion}
\def\xxsourceexo{}
\def\xxactivite{Colle 01}
\def\xxpartie{Exercices de colles}
\def\xxnumpartie{RdM}
\def\discipline{Sciences \\Industrielles de \\ l'Ingénieur}
\def\classe{\textsf{PT$\star$ -- PT}}
\def\xxauteur{\textsl{Xavier Pessoles}}

\def\xxtete{Sciences Industrielles de l'Ingénieur}

\def\xxpied{%
\xxpartie\\
\xxchapitre%
}
\begin{document}

\def\xxcompetences{}
\def\xxfigures{}


\setlength{\columnseprule}{.1pt}



%% COLLE 1
\livrettrue\def\xxactivite{Colle 01}
\proffalse
\newpage
\input{\repStyle/pagegarde_TD}
\vspace{4cm}
\pagestyle{fancy}
\thispagestyle{plain}


\begin{multicols}{2}
\section*{Cours}
En flexion, donner la forme du torseur de cohésion, l'expression de la contrainte et l'équation de la déformée. Vous vous appuierez sur un schéma dans chacun des cas. 

Donner le moment quadratique par rapport à l'axe $\axe{G}{z}$ d'une poutre de section carrée.


\renewcommand{\repExo}{522_RdM}
\graphicspath{{\repRel/Style/png/}{\repRel/ExercicesCompetences/C2_MettreEnOeuvreDemarche/C2_10_RdM_Cohesion/\repExo/images/}}
\input{\repRel/ExercicesCompetences/C2_MettreEnOeuvreDemarche/C2_10_RdM_Cohesion/\repExo/\repExo.tex}

\renewcommand{\repExo}{523_RdM}
\graphicspath{{\repRel/Style/png/}{\repRel/ExercicesCompetences/C2_MettreEnOeuvreDemarche/C2_10_RdM_Cohesion/\repExo/images/}}
\input{\repRel/ExercicesCompetences/C2_MettreEnOeuvreDemarche/C2_10_RdM_Cohesion/\repExo/\repExo.tex}

\renewcommand{\repExo}{528_BrocheFraisage}
\graphicspath{{\repRel/Style/png/}{\repRel/ExercicesCompetences/C2_MettreEnOeuvreDemarche/C2_10_RdM_Cohesion/\repExo/images/}}
\input{\repRel/ExercicesCompetences/C2_MettreEnOeuvreDemarche/C2_10_RdM_Cohesion/\repExo/\repExo.tex}
\end{multicols}


Lien vers le corrigé : 

\qrcode{https://github.com/xpessoles/2022_2023_Enseignements/raw/main/PT/02_RdM/Colle_02_RdM_Cor.pdf}

%% COLLE 2
\livrettrue\def\xxactivite{Colle 02}
\newpage
\input{\repStyle/pagegarde_TD}
\vspace{4cm}
\pagestyle{fancy}
\thispagestyle{plain}

\begin{multicols}{2}
\section*{Cours}
En torsion, donner la forme du torseur de cohésion, l'expression de la contrainte et l'équation de la déformation angulaire. Vous vous appuierez sur un schéma dans chacun des cas. 

Donner le moment quadratique par rapport à l'axe $\axe{G}{z}$ d'une poutre de section circulaire.

\renewcommand{\repExo}{524_RdM}
\graphicspath{{\repRel/Style/png/}{\repRel/ExercicesCompetences/C2_MettreEnOeuvreDemarche/C2_10_RdM_Cohesion/\repExo/images/}}
\input{\repRel/ExercicesCompetences/C2_MettreEnOeuvreDemarche/C2_10_RdM_Cohesion/\repExo/\repExo.tex}

\renewcommand{\repExo}{525_RdM}
\graphicspath{{\repRel/Style/png/}{\repRel/ExercicesCompetences/C2_MettreEnOeuvreDemarche/C2_10_RdM_Cohesion/\repExo/images/}}
\input{\repRel/ExercicesCompetences/C2_MettreEnOeuvreDemarche/C2_10_RdM_Cohesion/\repExo/\repExo.tex}

\renewcommand{\repExo}{529_Passerelle}
\graphicspath{{\repRel/Style/png/}{\repRel/ExercicesCompetences/C2_MettreEnOeuvreDemarche/C2_10_RdM_Cohesion/\repExo/images/}}
\input{\repRel/ExercicesCompetences/C2_MettreEnOeuvreDemarche/C2_10_RdM_Cohesion/\repExo/\repExo.tex}
\end{multicols}

Lien vers le corrigé : 

\qrcode{https://github.com/xpessoles/2022_2023_Enseignements/raw/main/PT/02_RdM/Colle_02_RdM_Cor.pdf}

%% COLLE 3
\livrettrue\def\xxactivite{Colle 03}
\newpage
\input{\repStyle/pagegarde_TD}
\vspace{4cm}
\pagestyle{fancy}
\thispagestyle{plain}

\begin{multicols}{2}
\section*{Cours}
En traction, donner la forme du torseur de cohésion, l'expression de la contrainte et la déformation relative. Vous vous appuierez sur un schéma dans chacun des cas. 

Donner le moment quadratique par rapport à l'axe $\axe{G}{z}$ d'une poutre de section rectangulaire (hauteur $h$ et largeur $b$).

\renewcommand{\repExo}{526_RdM}
\graphicspath{{\repRel/Style/png/}{\repRel/ExercicesCompetences/C2_MettreEnOeuvreDemarche/C2_10_RdM_Cohesion/\repExo/images/}}
\input{\repRel/ExercicesCompetences/C2_MettreEnOeuvreDemarche/C2_10_RdM_Cohesion/\repExo/\repExo.tex}

\renewcommand{\repExo}{527_RdM}
\graphicspath{{\repRel/Style/png/}{\repRel/ExercicesCompetences/C2_MettreEnOeuvreDemarche/C2_10_RdM_Cohesion/\repExo/images/}}
\input{\repRel/ExercicesCompetences/C2_MettreEnOeuvreDemarche/C2_10_RdM_Cohesion/\repExo/\repExo.tex}


\renewcommand{\repExo}{530_BancHelico}
\graphicspath{{\repRel/Style/png/}{\repRel/ExercicesCompetences/C2_MettreEnOeuvreDemarche/C2_10_RdM_Cohesion/\repExo/images/}}
\input{\repRel/ExercicesCompetences/C2_MettreEnOeuvreDemarche/C2_10_RdM_Cohesion/\repExo/\repExo.tex}
\end{multicols}

Lien vers le corrigé : 

\qrcode{https://github.com/xpessoles/2022_2023_Enseignements/raw/main/PT/02_RdM/Colle_02_RdM_Cor.pdf}
\end{document}



