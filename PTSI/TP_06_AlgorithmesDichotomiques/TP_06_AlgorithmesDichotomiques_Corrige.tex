% Load the kaobook class
\documentclass[
	fontsize=10pt, % Base font size
	twoside=true, % Use different layouts for even and odd pages (in particular, if twoside=true, the margin column will be always on the outside)
	%open=any, % If twoside=true, uncomment this to force new chapters to start on any page, not only on right (odd) pages
%%	secnumdepth=1, % How deep to number headings. Defaults to 1 (sections)
]{kaobook}

\newcommand{\discipline}{Informatique -- PTSI}
\newcommand{\auteur}{Xavier Pessoles}
\renewcommand{\auteur}{I. Cotta, X. Pessoles, V. Reydellet}
\newcommand{\repRel}{../../..}

\input{\repRel/Style/packages_v2}
\newcommand{\repStyle}{\repRel/Style}
\input{\repRel/Style/macros_SII_v2}
\input{\repRel/Style/environment_v2}
\usepackage{\repStyle/UPSTI_pedagogique}



\frenchsetup{StandardItemLabels=true}


% Load packages for testing
%\usepackage{blindtext}
%\usepackage{showframe} % Uncomment to show boxes around the text area, margin, header and footer
%\usepackage{showlabels} % Uncomment to output the content of \label commands to the document where they are used

% Load the bibliography package
%\usepackage{kaobiblio}
%\addbibresource{Cy_01_ModelisationSystemes.bib} % Bibliography file

% Load mathematical packages for theorems and related environments
\usepackage{kaotheorems}

% Load the package for hyperreferences
\usepackage{kaorefs}
%\usepackage{lmodern}

\graphicspath{{images/}{./}} % Paths where images are looked for

\makeindex[columns=3, title=Alphabetical Index, intoc] % Make LaTeX produce the files required to compile the index


\begin{document}

%%----------------------------------------------------------------------------------------
%%	BOOK INFORMATION
%%----------------------------------------------------------------------------------------

\titlehead{Xavier Pessoles - Informatique en PTSI}
\title[Xavier Pessoles - Informatique en PTSI]{Xavier Pessoles - Informatique en PTSI}
\author[XP]{Xavier Pessoles}
\date{\today}


\begingroup % Local scope for the following commands
\setstretch{1} % Uncomment to modify line spacing in the ToC
%\tableofcontents % Output the table of contents
\endgroup

%%----------------------------------------------------------------------------------------
%%	MAIN BODY
%%----------------------------------------------------------------------------------------
%
\mainmatter % Denotes the start of the main document content, resets page numbering and uses arabic numbers
%\setchapterstyle{kao} % Choose the default chapter heading style
%
%\chapter{First Chapter}
%
%\blindtext
%
%\pagelayout{wide} % No margins
%\addpart{Title of the Part}
%\pagelayout{margin} % Restore margins
%
%\chapter{Second Chapter}

\setchapterstyle{kao}

\setcounter{margintocdepth}{\sectiontocdepth}
\marginlayout
\graphicspath{{\repStyle/png}}

\pagestyle{xp.scrheadings}
%\pagestyle{xp.test}


%\input{\repStyle/Cycle_01_Entete}



%%%%%
\newcommand{\repExo}{Application_01_ROV}
\newcommand{\nomExo}{Application_01_ROV}
%\newcounter{cptApplication}[chapter]
%\newcounter{cptTD}[chapter]
\AtBeginEnvironment{corrige}{\small}

\livrettrue % Livrettrue :  eleve sans les corrections
\livrettrue
\colletrue
%%%%%


% ========= TD ========= 
\renewcommand{\repExo}{\repRel/Informatique/Exercices/S1_04_AlgorithmesDichotomiques/}
\renewcommand{\nomExo}{01_RechercheDichotomique}
\graphicspath{{\repStyle/png}{\repExo\nomExo/images}}


%\setchapterimage{fig_00.jpg}
\chapter*{TP 6 \\ 
Algorithmes Dichotomiques -- \ifprof Corrigé \else Sujet \fi}
\addcontentsline{toc}{section}{TP 6 : 
Algorithmes Dichotomiques -- \ifprof Corrigé \else Sujet \fi}

\iflivret \stepcounter{cptApplication} \else
\ifprof  \stepcounter{cptApplication} \else \fi
\fi


\input{\repExo\nomExo}

\input{\repExo/02_RechercheZero}

\input{\repExo/03_ExponentiationRapide}

\newpage
\input{\repExo/01_RechercheDichotomique-cor}

\input{\repExo/02_RechercheZero-cor}

\input{\repExo/03_ExponentiationRapide-cor}


%
%\proffalse
%\input{\repExo\nomExo/\nomExo} 
%\iflivret \else \proftrue
%\input{\repExo\nomExo/\nomExo}
%\fi
% ==========================





\end{document}
